\documentclass[11pt,a4paper]{article}

\usepackage{amsmath}
\usepackage{amsfonts}
\usepackage{amssymb}
\usepackage{stmaryrd}
\usepackage[utf8]{inputenc}
\usepackage[ngerman]{babel}
\usepackage{hyperref}
\usepackage{graphicx}
\usepackage{listings}
\usepackage{fancyhdr}
%\usepackage{fullpage}

\title{\textbf{Mathe Hausaufgaben zum 03. und 04. November 2016}}
\author{Matz Radloff(6946325)}

\pagestyle{fancy}
\fancyhf{}
\rhead{Matz Radloff(6946325) - Mathe I\_Inf}
\cfoot{\thepage}


\begin{document}
\pagenumbering{gobble}
  \maketitle
  \date{}
  \tableofcontents
  \newpage
  \pagenumbering{arabic}

\section{}

\subsection{(a)}
$A = \{1,2,3\}, B = \{a,b\}$

\subsubsection{i.}
Es gibt keine injektive Funktion $f : A \rightarrow B$, da $B$ weniger Elemente als $A$ enthält.

\subsubsection{ii.}
Ja, es gibt mehrere surjektive Funktionen $g : A \rightarrow B$, die nicht injektiv sind, z.B.:

\begin{table}[!th]
\begin{tabular}{|c|c|}
\hline
$x$ & $g(x)$\\
\hline
1 & a \\
2 & a \\
3 & b \\
\hline
\end{tabular}
\caption{Wertetabelle für g}
\label{ex:table}
\end{table}

\subsubsection{iii.}
Da es keine injektive Funktion gibt, die $A$ auf $B$ abbildet, existiert folglich auch keine Bijektion.


\subsection{(b)}
$A = \{1,2,3\}, B = \{a,b,c\}$

\subsubsection{i.}
Es gibt zwar eine injektive Funktion $f : A \rightarrow B$, welche allerdings auch surjektiv ist, da $B$ genauso viele Elemente wie $A$ enthält.

\subsubsection{ii.}
Es gibt keine surjektive Funktion $g : A \rightarrow B$, die nicht auch gleichzeitig injektiv ist, da sonst nicht allen Elementen aus $A$ ein Bildwert zugeordnet würde.

\newpage
\subsubsection{iii.}
Ja, es gibt folgende Bijektion:

\begin{table}[!th]
\begin{tabular}{|c|c|}
\hline
$x$ & $h(x)$\\
\hline
1 & a \\
2 & b \\
3 & c \\
\hline
\end{tabular}
\caption{Wertetabelle für g}
\label{ex:table}
\end{table}


\subsection{(c)}
$A = \{1,2,3\}, B = \{a,b,c,d\}$

\subsubsection{i.}
Ja, es gibt folgende injektive Funktion, welche nicht surjektiv ist:

\begin{table}[!th]
\begin{tabular}{|c|c|}
\hline
$x$ & $f(x)$\\
\hline
1 & a \\
2 & b \\
3 & c \\
\hline
\end{tabular}
\caption{Wertetabelle für g}
\label{ex:table}
\end{table}

\subsubsection{ii.}
Es gibt keine surjektive Funktion $g : A \rightarrow B$, weil $B$ mehr Elemente als $A$ enthält und bei Abbildung auf alle Elemente aus $B$, $g$ keine Funktion mehr wäre, da ein Elemente aus $A$ auf zwei aus $B$ abbilden müsste.

\subsubsection{iii.}
Da es keine surjektive Funktion gibt, existiert auch keine Bijektion.


% 2
\section{}

\subsection{(a)}
$f : \mathbb{Z} \rightarrow \mathbb{Z}; n \rightarrow -3n$

\paragraph{Injektivität}
$f(x_1) = f(x_2) \rightarrow -3x_1 = -3x_2 \rightarrow x_1 = x_2 \checkmark$

\paragraph{Surjektivität}
$x \in \mathbb{Z}; f(x) = y \rightarrow y = -3x \rightarrow x = -\frac{y}{3} \lightning$
Es liegt keine Surjektivität vor, da z.B. für $y = 2$ kein Urbild in $\mathbb{Z}$ existiert.

Die Funktion $f$ ist nicht bijektiv, da keine Surjektivität vorliegt.

\subsection{(b)}
$g : \mathbb{Z} \rightarrow \mathbb{Z}; n \rightarrow -2n+3$

\paragraph{Injektivität}
$g(x_1) = g(x_2) \rightarrow -2x_1+3 = -2x_2+3 \checkmark$

\paragraph{Surjektivität}
$g(x) = y \rightarrow y = -2x+3 \rightarrow x = \frac{3}{2}y \lightning$
Es liegt keine Surjektivität vor, da z.B. für $y = 2$ kein Urbild in $\mathbb{Z}$ existiert.

Folglich ist $g$ auch nicht bijektiv.


\subsection{(c)}
$h : \mathbb{Z} \rightarrow \mathbb{Z}; n \rightarrow n^2-3$

\paragraph{Injektivität}
$h(x_1) = h(x_2) \rightarrow x_1^2-3 = x_2^2-3 \rightarrow \pm x_1 = \pm _x2 \lightning$
$h$ ist nicht injektiv, da z.B. $f(x_1) = f(-x_2) gilt.$

\paragraph{Surjektivität}
$h(x) = y \rightarrow y = x^2-3 \rightarrow x = \sqrt{y+3} \lightning$
$h$ ist auch nicht surjektiv, da z.B. für $y=2$ kein Urbild existiert ($\sqrt{5} \notin \mathbb{Z}$).

$h$ ist also auch nicht bijektiv.


% 3
\section{}

\subsection{(a)}
$f : \mathbb{Z} \rightarrow \mathbb{Z} \times \mathbb{Z}; m \rightarrow (m^2-3, (m-2)^2)$

\paragraph{Injektivität}
\begin{align*}
f(x_1) &= f(x_2)\\
x_1^2-3 &= x_2^2-3 \rightarrow \pm x_1 = \pm x_2 \lightning
\end{align*}
Da schon für das erste Element die Injektivität nicht gilt, ist $f$ auch insgesamt nicht injektiv.

\paragraph{Surjektivität}
\begin{align*}
f(x) = (y_1,y_2)\\
y_1 = x_1^2-3 \rightarrow x_1 = \sqrt{y_1-3} \lightning\\
\end{align*}
$f$ ist nicht surjektiv und damit insgesamt auch nicht bijektiv.


\subsection{(b)}

$g: \mathbb{Z} \times \mathbb{Z} \rightarrow \mathbb{Z}; (n,m) \rightarrow -m-n^2$

\paragraph{Injektivität}
\begin{align*}
g(x_1, z_1) &= g(x_2, z_2)\\
-x_1-z_1^2 &= -x_2-z_2^2
\end{align*}
Annahme: $x_1 = x_2$
\begin{align*}
\pm z_1 &= \pm z_2 \lightning
\end{align*}
$\rightarrow$ Nicht injektiv.

\paragraph{Surjektivität}
\begin{align*}
g(x, z) &= y\\
-x-z^2 &= y\\
\end{align*}

\begin{align}
x &= z^2-y\\
z &= \sqrt{x-y}\\
z &= \sqrt{z^2-2y}\\
z^2 &= x^2-2y \lightning
\end{align}

$g$ ist nicht surjektiv und folglich auch nicht bijektiv.


\subsection{(c)}
$h: \mathbb{Z} \times \mathbb{Z} \rightarrow \mathbb{Z} \times \mathbb{Z}; (m,n) \rightarrow (m-n, m+n)$


% 4
\section{}
Für alle $n \in \mathbb{N}$ soll gelten:
\begin{equation*}
\sum_{i=1}^{n} (2i-1) = n^2
\end{equation*}

Induktionsstart: $n=1$
\begin{equation*}
(2-1) = 1^2 \rightarrow 1 = 1 \checkmark
\end{equation*}

Induktionsschritt:
\begin{align*}
\sum_{i=1}^{n+1} (2i-1) = (n+1)^2\\
\sum_{i=1}^{n} (2i-1) + (2(n+1)-1) = (n+1)^2\\
n^2 + 2n + 1 = n^2 + 2n + 1 \checkmark
\end{align*}
\end{document}