\documentclass[11pt,a4paper]{article}

\usepackage{amsmath}
\usepackage{amsfonts}
\usepackage{amssymb}
\usepackage{stmaryrd}
\usepackage[utf8]{inputenc}
\usepackage[ngerman]{babel}
\usepackage{hyperref}
\usepackage{graphicx}
\usepackage{listings}
\usepackage{fancyhdr}
%\usepackage{fullpage}

\title{\textbf{Mathe Hausaufgaben zum 03. und 04. November 2016}}
\author{Matz Radloff(6946325)}

\pagestyle{fancy}
\fancyhf{}
\rhead{Matz Radloff(6946325) - Mathe I\_Inf}
\cfoot{\thepage}


\begin{document}
\pagenumbering{gobble}
  \maketitle
  \date{}
  \tableofcontents
  \newpage
  \pagenumbering{arabic}

\section{}

\subsection{(a)}
$A = \{1,2,3\}, B = \{a,b\}$

\subsubsection{i.}
Es gibt keine injektive Funktion $f : A \rightarrow B$, da $B$ weniger Elemente als $A$ enthält.

\subsubsection{ii.}
Ja, es gibt mehrere surjektive Funktionen $g : A \rightarrow B$, die nicht injektiv sind, z.B.:

\begin{table}[!th]
\begin{tabular}{|c|c|}
\hline
$x$ & $g(x)$\\
\hline
1 & a \\
2 & a \\
3 & b \\
\hline
\end{tabular}
\caption{Wertetabelle für g}
\label{ex:table}
\end{table}

\subsubsection{iii.}
Da es keine injektive Funktion gibt, die $A$ auf $B$ abbildet, existiert folglich auch keine Bijektion.


\subsection{(b)}
$A = \{1,2,3\}, B = \{a,b,c\}$

\subsubsection{i.}
Es gibt zwar eine injektive Funktion $f : A \rightarrow B$, welche allerdings auch surjektiv ist, da $B$ genauso viele Elemente wie $A$ enthält.

\subsubsection{ii.}
Es gibt keine surjektive Funktion $g : A \rightarrow B$, die nicht auch gleichzeitig injektiv ist, da sonst nicht allen Elementen aus $A$ ein Bildwert zugeordnet würde.

\newpage
\subsubsection{iii.}
Ja, es gibt folgende Bijektion:

\begin{table}[!th]
\begin{tabular}{|c|c|}
\hline
$x$ & $h(x)$\\
\hline
1 & a \\
2 & b \\
3 & c \\
\hline
\end{tabular}
\caption{Wertetabelle für g}
\label{ex:table}
\end{table}


\subsection{(c)}
$A = \{1,2,3\}, B = \{a,b,c,d\}$

\subsubsection{i.}
Ja, es gibt folgende injektive Funktion, welche nicht surjektiv ist:

\begin{table}[!th]
\begin{tabular}{|c|c|}
\hline
$x$ & $f(x)$\\
\hline
1 & a \\
2 & b \\
3 & c \\
\hline
\end{tabular}
\caption{Wertetabelle für g}
\label{ex:table}
\end{table}

\subsubsection{ii.}
Es gibt keine surjektive Funktion $g : A \rightarrow B$, weil $B$ mehr Elemente als $A$ enthält und bei Abbildung auf alle Elemente aus $B$, $g$ keine Funktion mehr wäre, da ein Elemente aus $A$ auf zwei aus $B$ abbilden müsste.

\subsubsection{iii.}
Da es keine surjektive Funktion gibt, existiert auch keine Bijektion.


% 2
\section{}

\subsection{(a)}
$f : \mathbb{Z} \rightarrow \mathbb{Z}; n \rightarrow -3n$

\paragraph{Injektivität}
$f(x_1) = f(x_2) \rightarrow -3x_1 = -3x_2 \rightarrow x_1 = x_2 \checkmark$

\paragraph{Surjektivität}
$x \in \mathbb{Z}; f(x) = y \rightarrow y = -3x \rightarrow x = -\frac{y}{3} \lightning$
Es liegt keine Surjektivität vor, da z.B. für $y = 2$ kein Urbild existiert.


\end{document}