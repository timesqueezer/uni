\documentclass[11pt,a4paper]{article}

\usepackage{amsmath}
\usepackage{amsfonts}
\usepackage[utf8]{inputenc}
\usepackage[ngerman]{babel}
\usepackage{hyperref}
\usepackage{graphicx}
\usepackage{listings}
\usepackage{fancyhdr}
%\usepackage{fullpage}

\title{\textbf{Mathe Hausaufgaben zum 27. und 28. Oktober 2016}}
\author{Matz Radloff}

\pagestyle{fancy}
\fancyhf{}
\rhead{Matz Radloff - Mathe I\_Inf}
\cfoot{\thepage}


\begin{document}
\pagenumbering{gobble}
  \maketitle
  \date{}
  \tableofcontents
  \newpage
  \pagenumbering{arabic}

\section{}

\begin{align}
A &:= \{n \in \mathbb{N} : n > 3\}\\
B &:= \{n \in \mathbb{N} : n \mbox{ ist durch 14 teilbar}\}\\
C &:= \{n \in \mathbb{N} : n > 5, \mbox{ist durch 7 teilbar und ist gerade}\}
\end{align}

% (a)
\subsection[(a)]{(a) $A \subseteq B$}

Bedingung: $\forall a \in A : a \in B$

\begin{align}
a_{1} &= k+3, k \in \mathbb{N}\\
b_{1} &= 14l, l \in \mathbb{N}
\end{align}

Beweis durch Kontraposition: $k=1$ einsetzen, $a_{1} = b_{1}$ setzen:

\begin{align}
a_{1} &= 4\\
l &= \frac{2}{7}
\end{align}

$\rightarrow \exists a \in A : a \notin B$\\
Folglich ist die Aussage $A \subseteq B$ widerlegt, da $l$ keine natürliche Zahl ist. Es gibt also ein Element in A, dass nicht in B liegt.

% (b)
\subsection[(b)]{(b) $B \subseteq A$}

Bedingung: $\forall b \in B : b \in A$

\begin{align}
b_{1} &= 14l, l \in \mathbb{N}\\
a_{1} &= k+3, k \in \mathbb{N}
\end{align}

\textbf{Induktionsbeweis}

kleinstes $l$ einsetzen:
\begin{align}
l &= 1, b_{1} = a_{1}\\
k &= 11 \surd
\end{align}

generischen Fall prüfen:
\begin{align}
l &= n+1\\
14n+1 &= k+3\\
k &= 14n-2 \surd
\end{align}

Alle Elemente in B sind auch in A enthalten, also größer als 3.

% (c)
\subsection[(b)]{(c) $C \subseteq A$}
Bedingung: $\forall c \in C : c \in A$\\

\textbf{Direktbeweis}
Die kleinstmögliche Zahl $c \in C$ ist $14$. Da $14 > 3$ und $C$ außer $14$ nur größere Zahlen enthält gilt $C \subseteq A$.


% (d)
\subsection[(c)]{(c) $B=C$}
Bedingung: $B \subseteq C \wedge C \subseteq B$\\

$\forall c \in C : c=7o \wedge c=2p : o, p \in \mathbb{N} $\\

Zusammengefasst ergibt sich, dass alle Elemente aus $C$ - genau wie in $B$ - durch 14 teilbar sein müssen. Wenn man nur das kleinstmögliche $b$ ($14$) bildet und überprüft, dass es die verbleibende Bedingung $c > 5$ ($14 > 5\surd$) erfüllt, ist $B=C$ bewiesen.

% 2
\section{}

\begin{tabular}{llll}
\textbf{(a)} xor & \textbf{(b)} $\vee$ & \textbf{(c)} xor & \textbf{(d)} xor
\end{tabular}


% 3
\section{}
$\overline{A \cap B} = \overline{A} \cup \overline{B}$\\
Äquivalente Aussageform:
$\lnot(a \wedge b) = \lnot a \vee \lnot b$

\begin{table}[!th]
\begin{tabular}{|l|c|c|r|r|r|r|}
\hline
a & b & $a \wedge b$ & $\lnot(a \wedge b)$ & $\lnot a$ & $\lnot b$ & $\lnot a \vee \lnot b$ \\
\hline
0 & 0 & 0 & 1 & 0 & 1 & 1 \\
0 & 1 & 0 & 1 & 0 & 0 & 1 \\
1 & 0 & 0 & 1 & 1 & 1 & 1 \\
1 & 1 & 1 & 0 & 1 & 0 & 0 \\
\hline
\end{tabular}
\caption{Wahrheitstafel}
\label{ex:table}
\end{table}

Da die Spalten 4 und 7 die gleichen Werte enthalten, stimme die ursprüngliche Aussage.

%4
\section{}

\begin{align}
M &= \{a,b,c\}\\
\wp(M) &= \{\emptyset, \{a\}, \{b\}, \{c\}, \{a,b\}, \{a,c\}, \{b,c\}, \{a,b,c\}\}
\end{align}

% 5
\section{}

\subsection{(a)}
Im ersten Diagramm müsste dem Element 2 ein Funktionswert zugeordnet werden, damit die dargestellte Zuordnung eine Funktion darstellt. Im zweiten Diagramm dürfte das Element 2 nur einem anstatt zwei Werten zugeordnet werden.

\subsection{(b)}
Um eine injektive Funktion darzustellen müsste zusätzlich zu den Bedingungen aus \textbf{(a)} jeweils folgende Zuordnungen geändert werden:
\begin{itemize}
\item Im ersten Diagramm müssten entweder 4 oder 5 einem anderen Element der Zielmenge zugeordnet werden, damit nicht beide auf e abgebildet werden.
\item Im zweiten Diagramm müssten 3 oder 5 f zugeordnet werden.
\end{itemize}

\subsection{(c)}
Die Pfeile können nicht so abgeändert werden, dass surjektive Funktionen dargestellt werden, da B mehr Elemente als A enthält und somit die Bedingung, dass jedes Element der Zielmenge ein Urbild besitzt, nicht erfüllt werden kann.

\subsection{(d)}
\begin{equation}
f(4) = c
\end{equation}

\subsection{(e)}
\begin{equation}
f(4) = a
\end{equation}

\end{document}