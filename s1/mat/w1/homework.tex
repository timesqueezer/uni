\documentclass{article}

\usepackage{amsmath}
\usepackage{amsfonts}
\usepackage[utf8]{inputenc}
\usepackage[ngerman]{babel}
\usepackage{hyperref}
\usepackage{graphicx}
\usepackage{listings}

\title{\textbf{Mathe Hausaufgaben zum 27. und 28. Oktober 2016}}
\author{Matz Radloff}


\begin{document}
\pagenumbering{gobble}
  \maketitle
  \date{}
  \tableofcontents
  \newpage
  \pagenumbering{arabic}
\section{}

\begin{align}
A &:= \{n \in \mathbb{N} : n > 3\}\\
B &:= \{n \in \mathbb{N} : n = 14k, k \in \mathbb{N}\}\\
C &:= \{n \in \mathbb{N} : n > 5, \mbox{ist durch 7 teilbar und n ist gerade}\}
\end{align}

% a
\subsection[(a)]{(a) $A \subseteq B$}

Bedingung: $\forall a \in A : a \in B$

\begin{align}
a_{1} &= k+3, k \in \mathbb{N}\\
b_{1} &= 14l, l \in \mathbb{N}
\end{align}

kleinstmögliches $k$ einsetzen, $a_{1} = b_{1}$ setzen

\begin{align}
l &= \frac{2}{7}
\end{align}

Folglich ist die Aussage $A \subseteq B$ widerlegt, da $l$ keine natürliche Zahl ist. Es gibt also ein Element in A, dass nicht in B liegt.

% b
\subsection[(b)]{(b) $B \subseteq A$}

Bedingung: $\forall b \in B : b \in A$

\begin{align}
b_{1} &= 14l, l \in \mathbb{N}\\
a_{1} &= k+3, k \in \mathbb{N}
\end{align}

Induktionsbeweis:

\begin{align}
l &= 1, b_{1} = a_{1}\\
k &= 11 \surd\\
l &= n+1\\
14n+1 &= k+3\\
k &= 14n-2
\end{align}


\end{document}