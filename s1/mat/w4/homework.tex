\documentclass[11pt,a4paper]{article}

\usepackage{amsmath}
\usepackage{amsfonts}
\usepackage{amssymb}
\usepackage{stmaryrd}
\usepackage[utf8]{inputenc}
\usepackage[ngerman]{babel}
\usepackage{hyperref}
\usepackage{graphicx}
\usepackage{listings}
\usepackage{fancyhdr}
%\usepackage{fullpage}

\title{\textbf{Mathe Hausaufgaben zum 17. und 18. November 2016}}
\author{Matz Radloff(6946325), Elaha}

\pagestyle{fancy}
\fancyhf{}
\rhead{Matz Radloff(6946325), Elaha - Mathe I\_Inf}
\cfoot{\thepage}

\newcommand{\Myrightarrow}{\stackrel{\mathclap{\normalfont\mbox{(*)}}}{\Rightarrow}}


\begin{document}
\pagenumbering{arabic}
  \maketitle
  \date{}

\section{} % 1

\subsection*{(a)}

\paragraph{Behauptung:}
Ein Produkt $a \cdot b$ ganzer Zahlen ist nur dann durch $3$ teilbar, wenn mindestens einer der Faktoren durch $3$ teilbar ist.

\paragraph{Widerspruchsbeweis:}
Wir zeigen, dass das Produkt zweier, nicht durch $3$ teilbare Zahlen auch nicht durch $3$ teilbar sein kann.

\begin{equation*}
3k = a \cdot b; a,b \in \{3n+1, 3n+2\}; k,n \in \mathbb{Z}
\end{equation*}

Es ergeben sich die folgenden drei Fälle ($n, o, k \in \mathbb{Z}$):

\paragraph{$a = 3n+1, b = 3o+1$}

\begin{align*}
    3k &= (3n+1) \cdot (3o+1)\\
    3k &= 9no + 3n + 3o + 1\\
    3k &= 3(3no + n + o) + 1 \lightning
\end{align*}

\paragraph{$a = 3n+1, b = 3o+2$}

\begin{align*}
    3k &= (3n+1) \cdot (3o+2)\\
    3k &= 9no + 6n + 3o + 2\\
    3k &= 3(3no + 2n + o) + 2 \lightning
\end{align*}

\paragraph{$a = 3n+2, b = 3o+2$}
\begin{align*}
    3k &= (3n+2) \cdot (3o+2)\\
    3k &= 9no + 6n + 6o + 4\\
    3k &= 3(3no + 2n + 2o) + 4 \lightning
\end{align*}

Folglich gibt es keine Kombination von nicht durch $3$ teilbaren Zahlen, deren Produkt durch $3$ teilbar ist. \checkmark

\subsection*{(b)}
Wir zeigen, dass $\sqrt{3}$ eine irrationale Zahl ist.

\paragraph{Widerspruchsbeweis:}
Wir nehmen an, $\sqrt{3}$ ließe sich als Bruch $\frac{n}{m}; n, m \in \mathbb{Z}, m \neq 0$ darstellen, der maximal gekürzt sei.

Für den Beweis müssen wir vorher feststellen, ob die Wurzel einer durch $3$ teilbaren Zahl auch durch $3$ teilbar ist:

\begin{align*}
a, k &\in \mathbb{Z}\\
a &= 3k\\
a^2 &= 9k^2 = 3(3k^2) \tag*{\checkmark $(\ast)$}
\end{align*}

\begin{align*}
\sqrt{3} = \frac{n}{m} &\Rightarrow n^2 = 3m^2\\
3 \bigm| 3n^2 &\stackrel{(\ast)}{\Rightarrow} 3 \bigm| n\\
\Rightarrow 9 \bigm| n^2 &\Rightarrow 9k = n^2, k \in \mathbb{Z}\\
9k = 3m^2 &\Rightarrow 3k = m^2\\
\Rightarrow 3 \bigm| m^2 &\stackrel{(\ast)}{\Rightarrow} 3 \bigm| m\\
&\Rightarrow 3 \bigm| m, 3 \bigm| n \lightning
\end{align*}

\section{} % 2

\section{} % 3

\section{ggT(3213,234):} % 4
\begin{align*}
3213 &= 324 * 13 + 171\\
234 &= 171 * 1 + 63\\
171 &= 63 * 2 + 45\\
63 &= 45 * 1 + 18\\
45 &= 18 * 2 + 9\\
18 &= 9 * 2 + 0\\
\Rightarrow ggt(3213,234) &= 9
\end{align*}

\section{} % 5


\end{document}$