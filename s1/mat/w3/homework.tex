\documentclass[11pt,a4paper]{article}

\usepackage{amsmath}
\usepackage{amsfonts}
\usepackage{amssymb}
\usepackage{stmaryrd}
\usepackage[utf8]{inputenc}
\usepackage[ngerman]{babel}
\usepackage{hyperref}
\usepackage{graphicx}
\usepackage{listings}
\usepackage{fancyhdr}
%\usepackage{fullpage}

\title{\textbf{Mathe Hausaufgaben zum 10. und 11. November 2016}}
\author{Matz Radloff(6946325)}

\pagestyle{fancy}
\fancyhf{}
\rhead{Matz Radloff(6946325) - Mathe I\_Inf}
\cfoot{\thepage}


\begin{document}
\pagenumbering{arabic}
  \maketitle
  \date{}

\section{}

Für $n \in \mathbb{N}$ soll gelten:

\begin{equation*}
    A(n): 1 \cdot 2^1 + 2 \cdot 2^2 + 3 \cdot 2^3 + ... + n \cdot 2^n = (n-1) \cdot 2^{n+1} + 2
\end{equation*}

\subsection*{(a)} Formulierung der Gleichung $A(n)$ mit Hilfe des Summenzeichens:

\begin{equation*}
    \sum_{i=1}^{n}{i \cdot 2^i}
\end{equation*}

\subsection*{(b)} Prüfung, ob $A(n)$ für $n = 1, 2, 3$ richtig ist:

\begin{align*}
    A(1) &: 1 \cdot 2^1 = (1-1) \cdot 2^{1+1}+2 \Rightarrow 2 = 2 \tag*{\checkmark}\\
    A(2) &: 1 \cdot 2^1 + 2 \cdot 2^2 = (2-1) \cdot 2^{2+1}+2 \Rightarrow 10 = 10 \tag*{\checkmark}\\
    A(3) &: 1 \cdot 2^1 + 2 \cdot 2^2 + 3 \cdot 2^3 = (3-1) \cdot 2^{3+1}+2 \Rightarrow 34 = 34 \tag*{\checkmark}
\end{align*}

\subsection*{(c)} Beweis durch vollständige Induktion:

\paragraph{Induktionsanfang:}
Dass $A(1)$ gilt, wurde in (b) gezeigt.

\paragraph{Induktionsschritt:}
Es soll bewiesen werden, dass $A(n+1)$ unter Annahme, dass A(n) stimmt, gilt.

\begin{align*}
    \sum_{i=1}^{n+1}{i \cdot 2^i} &= (n+1-1) \cdot 2^{n+1+1}+2\\
    \sum_{i=1}^{n}{i \cdot 2^i} + (n+1) \cdot 2^{n+1} &= n \cdot 2^{n+2} + 2\\
    (n-1) \cdot 2^{n+1}+2 + (n+1) \cdot 2^{n+1} &= n \cdot 2^{n+2} + 2\\
    2n \cdot 2^{n+1} + 2 &= n \cdot 2^{n+2} + 2\\
    n \cdot 2^{n+2} +2 &= n \cdot 2^{n+2} + 2 \tag*{\checkmark}
\end{align*}

% 2
\section{} Für $n \in \mathbb{N}$ soll folgende Aussage gelten:

\begin{equation*}
    B(n): \sum_{i=1}^{n}{(2i-1)} = n^2
\end{equation*}

\subsection*{(a)} Prüfung, ob $B(n)$ für $n = 1, 2, 3$ gilt:

\begin{align*}
    B(1) &: (2 \cdot 1 - 1) = 1^2 \Rightarrow 1 = 1 \tag*{\checkmark}\\
    B(2) &: (2 \cdot 1 - 1) + (2 \cdot 2 - 1) = 2^2 \Rightarrow 4 = 4 \tag*{\checkmark}\\
    B(3) &: (2 \cdot 1 - 1) + (2 \cdot 2 - 1) + (2 \cdot 3 - 1) = 3^2 \Rightarrow 9 = 9 \tag*{\checkmark}
\end{align*}

\subsection*{(b)} Formulierung ohne das Summenzeichen:

\begin{equation*}
1+3+5+...+(2n-1) = n^2
\end{equation*}

\noindent
Die Summe der ersten $n$ ungeraden Zahlen lässt sich auch durch $n^2$ berechnen.

\subsection*{(c)} Beweis durch vollständige Induktion:

\paragraph{Induktionsanfang:}
Dass $B(1)$ gilt, wurde in (a) gezeigt.

\paragraph{Induktionsschritt:}
Es soll bewiesen werden, dass $B(n+1)$ unter Annahme, dass B(n) stimmt, gilt.

\begin{align*}
    \sum_{i=1}^{n+1}{(2i-1)} &= (n+1)^2\\
    \sum_{i=1}^{n}{(2i-1)} + (2 \cdot (n+1) - 1) &= (n+1)^2\\
    n^2 + 2n +1 &= (n+1)^2\\
    (n+1)^2 &= (n+1)^2 \tag*{\checkmark}
\end{align*}

% 3
\section{}

Für alle $n \in \mathbb{N}_0$ soll gelten:

\begin{equation*}
A(n): 7^n-1 = 6k; k \in \mathbb{N}_0
\end{equation*}

Beweis durch vollständige Induktion:

\paragraph{Induktionsanfang: A(1)}
\begin{align*}
7^1-1 &= 6k\\
k &= 1 \tag*{\checkmark}
\end{align*}

\paragraph{Induktionsschritt: A(n+1)}
\begin{align*}
7^{n+1}-1 &= 6p; p \in \mathbb{N}_0\\
7^n \cdot 7 - 1 &= 6p\\
7^n \cdot 6 + 7^n - 1 &= 6p\\
7^n \cdot 6 + 6k &= 6p \tag*{\checkmark}
\end{align*}
Da alle Elemente der letzten Gleichung durch $6$ teilbar ist, stimmt die Aussage.

% 4
\section{}
Es ist herauszufinden, für welche natürlichen Zahlen die Ungleichung $A(n): 2^n < n!$ gilt.

\paragraph{Vermutung:}

\begin{align*}
A(1): 2 &< 1 \lightning\\
A(2): 4 &< 2 \lightning\\
A(3): 8 &< 6 \lightning\\
A(4): 16 &< 24 \tag*{\checkmark}
\end{align*}
Vermutung: $A(n)$ gilt für alle $n > 3$.

\paragraph{Beweis durch vollständige Induktion:}

Für den Induktionsanfang wurde gezeigt, dass $A(4)$ wahr ist.

Induktionsschritt: A(n+1)
\begin{align*}
2^{n+1} &< (n+1)!\\
2^n \cdot 2 &< n! \cdot n+1\\
\mbox{nach I.A.:} 2^n \cdot 2 &< n! \cdot 2\\
\Rightarrow n! \cdot 2 &< n! \cdot n+1\\
\Rightarrow 2^n \cdot 2 < n! \cdot n+1 \tag*{\checkmark}
\end{align*}

% 5
\section{}
Damit $\thicksim$ eine Äquivalenzrelation auf der Menge $\mathcal{P}(\mathbb{N})$ sein kann, muss sie reflexiv, symmetrisch und transitiv sein.

\paragraph{Reflexivität:}
$A \rightarrow A$ ist eine bijektive Funktion. \checkmark

\paragraph{Symmetrie:}
Jede bijektive Funktion besitzt eine Umkehrfunktion, die auch bijektiv ist. Wenn $A \rightarrow B$ bijektiv ist, gibt es also auch eine bijektive Funktion $B \rightarrow A$.

\paragraph{Transitivität:}
Um die Bedingungen einer bijektiven Funktion zu erfüllen müssen Definitions- und Bildmenge gleichgroß sein. Da z.B. die Funktionen $A \rightarrow B$ und $B \rightarrow C$ beide $B$ enthalten, müssen sowohl $A$ als auch $C$ genauso groß sein wie $B$. Folglich sind auch $A$ und $C$ gleichgroß und es gibt eine Bijektion $A \rightarrow B$.

\end{document}$