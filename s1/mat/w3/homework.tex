\documentclass[11pt,a4paper]{article}

\usepackage{amsmath}
\usepackage{amsfonts}
\usepackage{amssymb}
\usepackage{stmaryrd}
\usepackage[utf8]{inputenc}
\usepackage[ngerman]{babel}
\usepackage{hyperref}
\usepackage{graphicx}
\usepackage{listings}
\usepackage{fancyhdr}
%\usepackage{fullpage}

\title{\textbf{Mathe Hausaufgaben zum 10. und 11. November 2016}}
\author{Matz Radloff(6946325)}

\pagestyle{fancy}
\fancyhf{}
\rhead{Matz Radloff(6946325) - Mathe I\_Inf}
\cfoot{\thepage}


\begin{document}
\pagenumbering{gobble}
  \maketitle
  \date{}
  \tableofcontents
  \newpage
  \pagenumbering{arabic}

\section{}

Für $n \in \mathbb{N}$ soll gelten:

\begin{equation*}
    A(n): 1 \cdot 2^1 + 2 \cdot 2^2 + 3 \cdot 2^3 + ... + n \cdot 2^n = (n-1) \cdot 2^{n+1} + 2
\end{equation*}

\subsection*{(a)} Formulierung der Gleichung $A(n)$ mit Hilfe des Summenzeichens:

\begin{equation*}
    \sum_{i=1}^{n}{i \cdot 2^i}
\end{equation*}

\subsection*{(b)} Prüfung, ob $A(n)$ für $n = 1, 2, 3$ richtig ist:

\begin{align*}
    A(1) &: 1 \cdot 2^1 = (1-1) \cdot 2^{1+1}+2 \Rightarrow 2 = 2 \tag*{\checkmark}\\
    A(2) &: 1 \cdot 2^1 + 2 \cdot 2^2 = (2-1) \cdot 2^{2+1}+2 \Rightarrow 10 = 10 \tag*{\checkmark}\\
    A(3) &: 1 \cdot 2^1 + 2 \cdot 2^2 + 3 \cdot 2^3 = (3-1) \cdot 2^{3+1}+2 \Rightarrow 34 = 34 \tag*{\checkmark}
\end{align*}

\subsection*{(c)} Beweis durch vollständige Induktion:

\paragraph{Induktionsanfang:}
Dass $A(1)$ gilt, wurde in (b) gezeigt.

\paragraph{Induktionsschritt:}
Es soll bewiesen werden, dass $A(n+1)$ unter Annahme, dass A(n) stimmt, gilt.

\begin{align*}
    \sum_{i=1}^{n+1}{i \cdot 2^i} &= (n+1-1) \cdot 2^{n+1+1}+2\\
    \sum_{i=1}^{n}{i \cdot 2^i} + (n+1) \cdot 2^{n+1} &= n \cdot 2^{n+2} + 2\\
    (n-1) \cdot 2^{n+1}+2 + (n+1) \cdot 2^{n+1} &= n \cdot 2^{n+2} + 2\\
    2n \cdot 2^{n+1} + 2 &= n \cdot 2^{n+2} + 2\\
    n \cdot 2^{n+2} +2 &= n \cdot 2^{n+2} + 2 \tag*{\checkmark}
\end{align*}

% 2
\section{} Für $n \in \mathbb{N}$ soll folgende Aussage gelten:

\begin{equation*}
    B(n): \sum_{i=1}^{n}{(2i-1)} = n^2
\end{equation*}

\subsection*{(a)} Prüfung, ob $B(n)$ für $n = 1, 2, 3$ gilt:

\begin{align*}
    B(1) &: (2 \cdot 1 - 1) = 1^2 \Rightarrow 1 = 1 \tag*{\checkmark}
    B(2) &: (2 \cdot 1 - 1) + (2 \cdot 2 - 1) = 2^2 \Rightarrow 4 = 4 \tag*{\checkmark}
    B(3) &: (2 \cdot 1 - 1) + (2 \cdot 2 - 1) + (2 \cdot 3 - 1) = 3^2 \Rightarrow 9 = 9 \tag*{\checkmark}
\end{align*}

\subsection*{(b)} Formulierung ohne das Summenzeichen:

$1+3+5+...+(2n-1) = n^2$

Die Summe der ersten $n$ ungeraden Zahlen lässt sich auch durch $n^2$ berechnen.

\subsection*{(c)} Beweis durch vollständige Induktion:

\paragraph{Induktionsanfang:}
Dass $B(1)$ gilt, wurde in (a) gezeigt.

\paragraph{Induktionsschritt:}
Es soll bewiesen werden, dass $B(n+1)$ unter Annahme, dass B(n) stimmt, gilt.

\begin{align*}
    \sum_{i=1}^{n+1}{(2i-1)} &= (n+1)^2\\
    \sum_{i=1}^{n}{(2i-1)} + (2 \cdot (n+1) - 1) &= (n+1)^2\\
    n^2 + 2n +1 &= (n+1)^2\\
    (n+1)^2 &= (n+1)^2 \tag*{\checkmark}
\end{align*}

% 3
\section{}
\end{document}