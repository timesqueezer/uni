\documentclass[11pt,a4paper]{article}
\usepackage{spreadtab}
\usepackage{amsmath}
\usepackage{amsfonts}
\usepackage[utf8]{inputenc}
\usepackage[ngerman]{babel}
\usepackage{hyperref}
\usepackage{graphicx}
\usepackage{listings}
\usepackage{fancyhdr}
\usepackage{eurosym}
\usepackage{url}
\usepackage{hyperref}
\usepackage[]{units}
%\usepackage{fullpage}

\title{\textbf{Rechnerstrukturen Hausaufgaben zum 09. November 2016}}
\author{Ali Ebrahimi Pourasad, Moritz Lahann, Matz Radloff}

\pagestyle{fancy}
\fancyhf{}
\lfoot{Ali Ebrahimi Pourasad(6948107)\\ Moritz Lahann(6948050)\\ Matz Radloff(6946325)}
\lhead{[RS\_GR1\_12]}
\cfoot{\thepage}


\begin{document}
  \maketitle
  \date{}
  Gruppe: [RS\_GR1\_12]

\section*{3.1} % 1
\label {sec:3.1}

\subsection*{(a) 00001101}
\begin{itemize}
  \item Ganze Zahlen: 13
  \item Betrag und Vorzeichen: 13
  \item Exzess-127 Kodierung: $13-127 = -114$
  \item Einerkomplement: 13
  \item Zweierkomplement: 13
 \end{itemize}

\subsection*{(b) 01100111}
\begin{itemize}
  \item Ganze Zahlen: 103
  \item Betrag und Vorzeichen: 103
  \item Exzess-127 Kodierung: $103-127 = -24$
  \item Einerkomplement: 103
  \item Zweierkomplement: 103
 \end{itemize}

\subsection*{(c) 10000101}
\begin{itemize}
  \item Ganze Zahlen: 133
  \item Betrag und Vorzeichen: -5
  \item Exzess-127 Kodierung: $133-127 = 6$
  \item Einerkomplement: $10000101 \rightarrow 11111010 = -122$
  \item Zweierkomplement: $10000101 \rightarrow 11111011 = -123$
 \end{itemize}

\subsection*{(d) 11111001}
\begin{itemize}
  \item Ganze Zahlen: 249
  \item Betrag und Vorzeichen: -121
  \item Exzess-127 Kodierung: $249-127 = 126$
  \item Einerkomplement: $11111001 \rightarrow 10000110 = -6$
  \item Zweierkomplement: $11111001 \rightarrow 10000111 = -7$
 \end{itemize}


\section*{3.2} % 1
\label {sec:3.2}

\subsection*{(a) $(56)_{10}$}
\begin{align*}
    56:2 &= 28 \rightarrow \mbox{R}0 \\
    28:2 &= 14 \rightarrow \mbox{R}0 \\
    14:2 &= 7  \rightarrow \mbox{R}0 \\
    7:2 &= 3   \rightarrow \mbox{R}1 \\
    3:2 &= 1   \rightarrow \mbox{R}1 \\
    1:2 &= 0   \rightarrow \mbox{R}1 \\
\end{align*}

$(56)_{10}=(00111000)_2$

\paragraph{Vorlesung:}
\begin{itemize}
    \item Einerkomplement: $(11000111)_2$
    \item Zweierkomplement: $(11001000)_2$
\end{itemize}

\subsection*{(b)}

Man kommt mit dem Algorithmus zu dem selben Ergebnis, da es äquivalent ist, das (b-1)-Komplement (Invertierung der Bits) von einer Zahl zu bilden und darauf 1 zu addieren, als wenn man die erste 1 stehen lässt und ab dann das Komplement bildet.

\newpage
\section*{3.3}
\label {sec:3.3}


$Z = (-56)_{10} = (11001000)_{K2}$ (s.3.2)

\paragraph{Dualdarstellung:}
$Z = (-56)_{10}$
\begin{align*}
-56: 2 = -28 \mbox{ Rest} &\rightarrow 0\\
-28: 2 = -14 \mbox{ Rest} &\rightarrow 0\\
-14 :2 = -7 \mbox{ Rest} &\rightarrow 0\\
-7 : 2 = -4 \mbox{ Rest} &\rightarrow 1\\
-4 : 2 = -2 \mbox{ Rest} &\rightarrow 0\\
-2 : 2 = -1 \mbox{ Rest} &\rightarrow 0\\
-1 : 2 = -1 \mbox{ Rest} &\rightarrow 1\\
-1 : 2 = -1 \mbox{ Rest} &\rightarrow 0 \uparrow \mbox{Leserichtung}
\end{align*}
Ab hier Überlauf

\section*{3.4}
\label {sec:3.4}

Um eine Subtraktion mit Hilfe von Komplementen durchzuführen, kann man das zu subtrahierende Element als 9-Komplement darstellen und mit der Addition von 1 in die 10-Komplement-Darstellung uberführen und schließlich zu der positiven Zahl addieren, wobei der Überlauf dann weggelassen wird, sodass eine Subtraktion stattfindet.

\subsection*{(a) $1385-532$}

\begin{align*}
(0532)_{10} = (9467)_{K9} &= (9468)_{K10}\\
1382 + 9468 &= (1)0850
\end{align*}
Durch Weglassen des Übertrags erhält man 850.


\subsection*{(b) $372-687$}

\begin{align*}
(0687)_{10} = (9312)_{K9} &= (9313)_{K10}\\
372 + 9313 = 9685 \rightarrow -315
\end{align*}

\subsection*{(c) $1385-532$}

$(1385)_{10}$:
\begin{align*}
    1385:2 &= 692 \rightarrow \mbox{R}1 \\
    692:2 &= 346 \rightarrow \mbox{R}0 \\
    346:2 &= 173  \rightarrow \mbox{R}0 \\
    173:2 &= 86   \rightarrow \mbox{R}1 \\
    86:2 &= 43   \rightarrow \mbox{R}0 \\
    43:2 &= 21   \rightarrow \mbox{R}1 \\
    21:2 &= 10   \rightarrow \mbox{R}1 \\
    10:2 &= 5   \rightarrow \mbox{R}0 \\
    5:2 &= 2   \rightarrow \mbox{R}1 \\
    2:2 &= 1   \rightarrow \mbox{R}0 \\
    1:2 &= 1   \rightarrow \mbox{R}1 \uparrow \mbox{Leserichtung}
\end{align*}
$(1385)_{10}=(010101101001)_2$\\

$(532)_{10}$:
\begin{align*}
    532:2 &= 266 \rightarrow \mbox{R}0 \\
    266:2 &= 133 \rightarrow \mbox{R}0 \\
    133:2 &= 66  \rightarrow \mbox{R}1 \\
    66:2 &= 33   \rightarrow \mbox{R}0 \\
    33:2 &= 16   \rightarrow \mbox{R}1 \\
    16:2 &= 8   \rightarrow \mbox{R}0 \\
    8:2 &= 4   \rightarrow \mbox{R}0 \\
    4:2 &= 2   \rightarrow \mbox{R}0 \\
    2:2 &= 1   \rightarrow \mbox{R}0 \\
    1:2 &= 1   \rightarrow \mbox{R}1 \uparrow \mbox{Leserichtung}
\end{align*}

$(532)_{10} = (001000010100)_2 = (110111101011)_{K1} = (110111101100)_{K2}$

\newpage
\subsection*{(d) $372-687$}

$(372)_{10}$:
\begin{align*}
    372:2 &= 186 \rightarrow \mbox{R}0 \\
    186:2 &= 93 \rightarrow \mbox{R}0 \\
    93:2 &= 46  \rightarrow \mbox{R}1 \\
    46:2 &= 23   \rightarrow \mbox{R}0 \\
    23:2 &= 11   \rightarrow \mbox{R}1 \\
    11:2 &= 5   \rightarrow \mbox{R}1 \\
    5:2 &= 2   \rightarrow \mbox{R}1 \\
    2:2 &= 1   \rightarrow \mbox{R}0 \\
    1:2 &= 1   \rightarrow \mbox{R}1 \uparrow \mbox{Leserichtung}
\end{align*}

$(372)_{10} = (000101110100)_2$\\

$(687)_{10}$:
\begin{align*}
    687:2 &= 343 \rightarrow \mbox{R}1 \\
    343:2 &= 171 \rightarrow \mbox{R}1 \\
    171:2 &= 85  \rightarrow \mbox{R}1 \\
    85:2 &= 42   \rightarrow \mbox{R}1 \\
    42:2 &= 21   \rightarrow \mbox{R}0 \\
    21:2 &= 10   \rightarrow \mbox{R}1 \\
    10:2 &= 5   \rightarrow \mbox{R}0 \\
    5:2 &= 2   \rightarrow \mbox{R}1 \\
    2:2 &= 1   \rightarrow \mbox{R}0 \\
    1:2 &= 1   \rightarrow \mbox{R}1 \uparrow \mbox{Leserichtung}
\end{align*}

$(687)_{10} = (001010101111)_2 = (110101010000)_{K1} = (110101010001)_{K2}$\\
$(111011000101)_2 = (000100111010)_{K1} = (000100111011)_{K2}$ \\

$Z = (000100111011)_2$:

$(000100111011)_2 = 2^{0} \cdot 1 + 2^{1} \cdot 1 + 2^{3} \cdot 1 + 2^{4} \cdot 1 + 2^{5} \cdot 1 + 2^{6} \cdot 1 + 2^{8} \cdot 1 = (315)_{10}$

Da kein Übertrag vorhanden war: $(-315)_10$

\newpage
\section*{3.5}
\label{sec:3.5}



\subsection*{(a)}
\begin{equation*}
(47,252 | 3)_{10} = 47,252 \cdot 10^3
\end{equation*}

Normalisiert:

\begin{equation*}
(4,7252 | 3+1 )_{10} = (4,7252 | 4 )_{10} = 4,7252 \cdot 10^4
\end{equation*}

\subsection*{(b)}
\begin{equation*}
(-10101,11 | -101)_2 = -10101,11*2^{-101}
\end{equation*}

Normalisiert:

\begin{equation*}
(-1,010111 | -101+100)_2 = (-1,010111 | -1001)_2 = -1,010111*1010^{-1001}
\end{equation*}

\subsection*{(c)}
\begin{equation*}
(-0,002DA | C)_{16} = -0,002DA*16^C
\end{equation*}

Normalisiert:

\begin{equation*}
(-2,DA |C-3)_{16}=(-2,DA |9)_{16}=-2,DA*A^9
\end{equation*}

\section*{3.6}
\label{sec:3.6}

\subsection*{(a) $(1011000)_2$}

\begin{itemize}
    \item Normalisiert: $1,011000 * 2^{110}$
    \item Vorzeichen: $0$
    \item Exponent(Exzess-127): $01111111 + 110 = 10000101 (133 - 127 = 6)$
    \item Mantisse: $011000$ $00000000000000000000000$
\end{itemize}

$\rightarrow$ IEEE754 $(0$ $10000101$ $011000$ $00000000000000000000000)$

\subsection*{(b) $(-10011011,101)_2$}

\begin{itemize}
    \item Normalisiert: $-1,0011011101 * 2^{111}$
    \item Vorzeichen: 1
    \item Exponent(Exzess-127): $01111111+111= 10000110(133 - 127 = 7)$
    \item Mantisse: $011000$ $00000000000000000000000$ \\
\end{itemize}

$\rightarrow$ IEEE754 $(1$ $10000110$ $0011011101$ $0000000000000000000$


\end{document}