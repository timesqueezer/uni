\documentclass[11pt,a4paper]{article}

\usepackage{amsmath}
\usepackage{amsfonts}
\usepackage[utf8]{inputenc}
\usepackage[ngerman]{babel}
\usepackage{hyperref}
\usepackage{graphicx}
\usepackage{listings}
\usepackage{fancyhdr}
\usepackage{url}
\usepackage{hyperref}
\usepackage[]{units}
%\usepackage{fullpage}

\title{\textbf{Rechnerstrukturen Hausaufgaben zum 26. Oktober 2016}}
\author{Ali Ebrahimi Pourasad, Moritz Lahann, Matz Radloff}

\pagestyle{fancy}
\fancyhf{}
\lfoot{Ali Ebrahimi Pourasad(6948107)\\ Moritz Lahann(6948050)\\ Matz Radloff(6946325)}
\lhead{[RS\_GR1\_12]}
\cfoot{\thepage}


\begin{document}
\pagenumbering{gobble}
  \maketitle
  \date{}
  \tableofcontents
  \newpage
  \pagenumbering{arabic}

\section{2} % 2

\subsection{2.1} % 2.1

\paragraph{Gegeben:}
\begin{itemize}
  \item Takt: 3,2GHz
  \item Max. 5 Oparationen parallel
  \item Leistungsverbrauch: 80W
\end{itemize}


\subsubsection{2.1 (a)}
\paragraph{Gegeben:}
\begin{itemize}
  \item i5, Haswell Chipgröße: $A_C = \unit[177]{mm^2} = \unit[1,77]{cm^2}$
  \item elektrische Herdplatte Leistung: $P = \unit[2]{KW} = \unit[2000]{W}$
  \item Durchmesser von Platte: $d = \unit[15]{cm}$
\end{itemize}

\paragraph{Rechnung:}
\paragraph{}i5 Leistungsdichte:
\begin{equation}
\frac{P}{A} = \frac{\unit[80]{W}}{\unit[1,77]{cm^2}} = \unitfrac[45,2]{W}{cm^2}
\end{equation}

\paragraph{}Fläche einer Herdplatte:
\begin{equation}
A_H = \pi\cdot(\frac{d}{2})^2 =  \pi\cdot\unit[56.25]{cm^2} = \unit[176,71]{cm^2}
\end{equation}

\paragraph{}Leistungsdichte einer Herdplatte:
\begin{equation}
\frac{P}{A} = \frac{\unit[2000]{W}}{\unit[176,71]{cm^2}} = \unitfrac[11,32]{W}{cm^2}
\end{equation}


\subsubsection{2.1 (b)}
\paragraph{Gegeben:}
Spannung: $U = \unit[3.7]{V}$
Ladung: $Q = \unit[3000]{mAh}$
Zeit: $t = \unit[3]{h}$

\paragraph{Rechnung:}
Um die Leistung $P = U \cdot I$ zu berechnen wird $I = Q/t$ substituiert:

\begin{equation}
P = U\cdot\frac{Q}{t} = \unit[3.7]{V} \cdot \unit[1]{A} = \unit[3.7]{W}
\end{equation}

Da die CPU die Hälfte der elektrischen Energie verbraucht ergibt sich:

\begin{align}
P_{CPU} = P/2 &= \unit[1.85]{W}\\
A_{Mobile-CPU} = \unit[16]{mm^2} &= \unit[0.16]{cm^2}
\end{align}


Leistungsdichte: $P/A = \frac{\unit[1.85]{W}}{\unit[0.016]{cm^2}} = \unitfrac[11.5625]{W}{cm^2}$
\iffalse
c)
n_{Desktop-CPU} = 3.2GHz * 5 (Operationen) = 1,6 * 10^10 Operationen / s
t_1 (Zeit pro Operation) = 1/n_{Desktop-CPU}
E = Pt = 80W * (1 / 1.6E10 Operationen/s)) = 5E-9 Ws = 5000 nJ / Operation

n_{Mobile_CPU} = 1G Operationen / s = 10^9 Operationen / s
t_2 = 1/n_{Mobile-CPU}
E = Pt = 1.85W * (10E-9 Operations / s) = 1850 nJ / Operation

kannst du die rechnung nochmal erklären/aufschlüsseln? Ich blick da nicht durch

Weiß jemand was zur zweiten Teilfrage? Ich geh sonst morgen nochmal die Folien der Vorlesung durch.

Zu Smart Dust: 
- Energieverbrauch steigt proportional zur Taktrate StrongArm1100 hat 133MHz bzw. 190MHz (http://datasheets.chipdb.org/Intel/STRONG/SA1100.HTM) vs 3.2GHz
- speziell für eine Aufgabe entwickelt --> effizienter, weniger Leistung benötigt vs. PC: schneller Allrounder
-

2.2

 Geg: 2022,562 MIlliarden € = c =  2022562000000€ = 202256200000000 cent
a)
x := Benötigte Stellen in Binär

2^x >= c, x \in N
x >= log_2(c)
x >= 40.87932 -> x = 41  ist nicht auf den cent genau!
log_2(202256200000000cent)=47,5 --> 48 stellen
48 ist richtig
b)
x := Stellen im Zahlensystem zur Basis 5
x >= log_5(c)
x >= 17.60576 -> x = 18 nicht auf den cent genau
cent: 21 Stellen
21 ist richitg

oh, ja hab ich nicht bedacht, ich kann ja sonst beides reinschreiben

2.3

a)

Z = (42)10

Dualdarstellung:
    
Z = (42)10:
42: 2 = 21 Rest    0     20
21: 2 = 10 Rest    1
10:2 = 5 Rest       0
5 : 2 = 2 Rest       1 
2 : 2 = 1 Rest        0
1 : 2 = 0 Rest        1    25 ↑ Leserichtung

Z = (42)10=  (101010)2

Oktaldarstellung:
Z= (101010)2 :
010 =  2
101=5
(Grupierungen von jeweils 3 Bits)

Z = (42)10=  (101010)2 = (52)8

Hexadezimaldarstellung:
Z= (101010)2:
1010= A
(00)10=2
(Grupierungen von jewels 4 Bits)

Z = (42)10=  (101010)2 = (52)8 = (2A)16

b)

Z = (1969)10

Dualdarstellung:
Z = (1969)10:
1969: 2 = 984 Rest    1     20
984: 2 = 492 Rest      0
492:2 = 246 Rest       0
246 : 2 = 123 Rest     0
123 : 2 = 61 Rest       1
61:2 = 30 Rest            1
30:2= 15 Rest             0
15 :2 = 7 Rest              1
7 :2 = 3 Rest               1
3:2 = 1 Rest                1 
1 : 2 = 0 Rest               1         210 ↑ Leserichtung

Z = (1969)10=  (11110110001)2

Oktaldarstellung:
Z=  (11110110001)2:
001= 1
110=6
110= 6
(0)11=3

(Grupierungen von jeweils 3 Bits)
Z = (1969)10=  (11110110001)2 = (3661)8

Hexadezimaldarstellung:
Z=(11110110001)2:
0001=1
1011=B
(0)111= 7

(Grupierungen von jewels 4 Bits)
Z = (1969)10=  (11110110001)2 = (3661)8 = (7B1)16

c)
Z= (5,5625)10

Dualdarstellung:
Z = (5,0)10:
5 : 2 = 2 Rest       1 
2 : 2 = 1 Rest        0
1 : 2 = 0 Rest        1    23 ↑ Leserichtung

(5)10= (101)2

(0,5625)10:
2 * 0,5625 = 1,125→ 1     2-1↓ Leserichtung
2 * 0,125 = 0,25 → 0
2 * 0,25 = 0,5 → 0
2 * 0,5 = 1,0 → 1

(0,5625)10 =(0,1001)2


Z = (5,5625)10=  (101,1001)2

Oktaldarstellung:
Z=  (101,0)2:
101= 5

Z=  (0,1001)2:
100=4
1(00)= 4

(Grupierungen von jeweils 3 Bits)

Z = (5,5625)10=  (101,1001)2 = (5,44)8

Hexadezimaldarstellung:
Z=(101,0)2:
(0)101= 5

Z=(0,1001)2:
1001=9

(Grupierungen von jewels 4 Bits)
Z = (5,5625)10=  (101,1001)2 = (5,44)8 = (5,9)16

d)
Z = (375,375)10


Dualdarstellung:
Z = (375,0)10:
375 : 2 = 187 Rest       1   20
187 : 2 = 93 Rest        1
93 : 2 = 46 Rest           1
46 : 2 = 23 Rest           0
23 : 2 = 11 Rest           1
11 : 2 = 5 Rest              1
5 : 2 = 2 Rest                1
2: 2 = 1  Rest                 0        
1 : 2 = 0 Rest                 1    29 ↑ Leserichtung

(375,0)10= (101110111)2

(0,375)10:
2 * 0,375 = 0,75→ 0     2-1↓ Leserichtung
2 * 0,75= 1,5 → 1
2 * 0,5 = 1,0 → 1

(0,375)10 =(0,011)2


Z = (375,375)10=  (101110111,011)2

Oktaldarstellung:
Z=  (101110111,0)2:
111=7
110=6
101= 5

Z=  (0,011)2:
011= 3

(Grupierungen von jeweils 3 Bits)

Z = (375,375)10=  (101110111,011)2 = (567,3)8

Hexadezimaldarstellung:

Z=  (101110111,0)2 :

0111=7
0111=7
(000)1=1


Z=  (0,011)2 :
011(0)= 6


(Grupierungen von jewels 4 Bits)
Z = (375,375)10=  (101110111,011)2 = (567,3)8 = ( 177,6)16

2.4a)
 a)
1*2^3 + 1*2^2 + 1*2^1 + 0*2^0 + 1*2^-1 + 0*2^-1 + 0*2^-2 + 1*2^-3
1*2^3 + 1*2^2 + 1*2^1 + 0*2^0 + 1*2^-1 + 0*2^-2 + 0*2^-3 + 1*2^-4 ist richtig. du hast zwei mal die 1

es sind 14,5625.
oh wow da war ich wohl etwas voreilig.

 b)
0-Koeffizienten weggelassen
1*2^4 + 1*2^2 + 1*2^0 + 1*2^-1 + 1*2^-2 + 1*2^-4 + 1*2^-5
= 21.84375 ist richtig.

2.5

(9A,C)_16 = 9*16^1 + 10*16^0 + 12*16^-1 = 154.75
stimmt

\fi

\newpage

\bibliography{wiki}
\bibliographystyle{ieeetr}

\end{document}