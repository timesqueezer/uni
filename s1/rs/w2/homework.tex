\documentclass[11pt,a4paper]{article}

\usepackage{amsmath}
\usepackage{amsfonts}
\usepackage[utf8]{inputenc}
\usepackage[ngerman]{babel}
\usepackage{hyperref}
\usepackage{graphicx}
\usepackage{listings}
\usepackage{fancyhdr}
\usepackage{eurosym}
\usepackage{url}
\usepackage{hyperref}
\usepackage[]{units}
%\usepackage{fullpage}

\title{\textbf{Rechnerstrukturen Hausaufgaben zum 02. November 2016}}
\author{Ali Ebrahimi Pourasad, Moritz Lahann, Matz Radloff}

\pagestyle{fancy}
\fancyhf{}
\lfoot{Ali Ebrahimi Pourasad(6948107)\\ Moritz Lahann(6948050)\\ Matz Radloff(6946325)}
\lhead{[RS\_GR1\_12]}
\cfoot{\thepage}


\begin{document}
\pagenumbering{gobble}
  \maketitle
  \date{}
  \tableofcontents
  \newpage
  \pagenumbering{arabic}

\section{2} % 2

\subsection{2.1} % 2.1

\paragraph{Gegeben:}
\begin{itemize}
  \item Takt: 3,2GHz
  \item Max. 5 Oparationen parallel
  \item Leistungsverbrauch: 80W
\end{itemize}


\subsubsection{2.1 (a)}
\paragraph{Gegeben:}
\begin{itemize}
  \item Core i5 (Haswell), Chipgröße: $A_C = \unit[177]{mm^2} = \unit[1,77]{cm^2}$
  \item elektrische Herdplatte Leistung: $P = \unit[2]{KW} = \unit[2000]{W}$
  \item Durchmesser von Platte: $d = \unit[15]{cm}$
\end{itemize}

\paragraph{Rechnung:}
\paragraph{}Leistungsdichte:
\begin{equation*}
\frac{P}{A_C} = \frac{\unit[80]{W}}{\unit[1,77]{cm^2}} = \unitfrac[45,2]{W}{cm^2}
\end{equation*}

\paragraph{}Fläche einer Herdplatte:
\begin{equation*}
A_H = \pi\cdot(\frac{d}{2})^2 =  \pi\cdot\unit[56.25]{cm^2} = \unit[176,71]{cm^2}
\end{equation*}

\paragraph{}Leistungsdichte einer Herdplatte:
\begin{equation*}
\frac{P}{A} = \frac{\unit[2000]{W}}{\unit[176,71]{cm^2}} = \unitfrac[11,32]{W}{cm^2}
\end{equation*}

Durch die sehr kleine Fläche des Prozessors hat dieser eine über viermal so große Leistungsdichte wie die der Herdplatte.

\newpage
\subsubsection{2.1 (b)}
\paragraph{Gegeben:}
\begin{align*}
\mbox{Spannung}: U &= \unit[3.7]{V}\\
\mbox{Ladung}: Q &= \unit[3000]{mAh}\\
\mbox{Zeit}: t &= \unit[3]{h}
\end{align*}

\paragraph{Rechnung:}
Um die Leistung $P = U \cdot I$ zu berechnen wird $I = Q/t$ substituiert:

\begin{equation*}
P = U\cdot\frac{Q}{t} = \unit[3.7]{V} \cdot \unit[1]{A} = \unit[3.7]{W}
\end{equation*}

Da die CPU die Hälfte der elektrischen Energie benötigt ergibt sich:

\begin{align*}
P_{CPU} = P/2 &= \unit[1.85]{W}\\
A_{Mobile-CPU} = \unit[16]{mm^2} &= \unit[0.16]{cm^2}
\end{align*}


Leistungsdichte: $P/A = \frac{\unit[1.85]{W}}{\unit[0.16]{cm^2}} = \unitfrac[11.5625]{W}{cm^2}$

\subsection{2.1 (c)}

\paragraph{}
Über die Anzahl der Operationen pro Sekunde $n_1$ (Desktop-CPU), $n_2$ (Mobile-CPU), bzw. der daraus foldenden Zeiten pro Operation $t_1, t_2$, lässt sich die benötigte Energie pro Operation mit $E = Pt$ errechnen.

\paragraph{Desktop}
\begin{align*}
n_1 &= \unit[3.2]{GHz} \cdot \unit[5]{Ops.} = \frac{1,6 \cdot \unit[10^{10}]{Ops.}}{\unit{s}}\\
t_1 &= 1/n_1\\
E_1 &= Pt_1 = \unit[80]{W} \cdot \unitfrac[1.6^{-10}]{Ops.}{s}\\
&= \unit[5^{-9}]{Ws} = \unit[5]{nJ}\\
\end{align*}

\paragraph{Mobile}
\begin{align*}
n_2 &= \unitfrac[10^9]{Op}{s}\\
t_2 &= 1/n_2\\
E_2 &= Pt_2 = \unit[1.85]{W} \cdot \unitfrac[10^{-9}]{Ops}{s} = \unit[1.85]{nJ}
\end{align*}

Für den Energieverbrauch pro Rechenoperation ergeben sich $\unit[5]{nJ}$ für die Desktop-CPU und $\unit[1.85]{nJ}$ für die Mobile-CPU.

Die Differenz zum in der Vorlesung angegebenen Wert von $\unitfrac[1]{nJ}{Op}$ des StrongArm1100 lässt dadurch erklären, dass dieser zum einen mit seiner 32-bit RISC Architektur und nur einem Kern effektiv weniger Transistoren schaltet, als eine Desktop-CPU. Vor allem im direkten Vergleich zu einem Core i5 (Haswell) zeigt sich, dass letzterer mit 1.5 Mrd. Transistoren fast 1000-mal so viele besitzt, wie der StrongArm mit 2.5 Mio.. Außerdem taktet er nur mit $\unit[133]{MHz}$ bzw. $\unit[190]{MHz}$, was zu einem deutlich niedrigeren Energieverbrauch führt, da sich dieser proportional zur Taktfrequenz verhält\cite{wiki:1} \cite{intel:1}.

\subsection{2.2}
\paragraph{Gegeben:}
$2022,562$ Mrd. \euro: $c = 2022562000000$ \euro\\
$= 202256200000000$ cent

\subsubsection{2.2 (a)}
\paragraph{Rechnung:}
$x :=$ Benötigte Stellen in Binär\\

Gesucht ist der 2er-Logarithmus der nächsthöheren 2er-Potenz.

\begin{align*}
2^x &>= c, x \in N\\
x &>= log_2(c)\\
x &>= 47.52 \rightarrow x = 48
\end{align*}

\paragraph{}
Für die Darstellung werden 48 bzw 49 bits, falls die Schulden als negative Zahl gespeichert werden sollen, benötigt.

\subsubsection{2.2 (b)}
\paragraph{Rechnung:}
$x := $ Stellen im Zahlensystem zur Basis $5$

\begin{align*}
x &\geq log_5(c)\\
x &\geq 20.47 \rightarrow x = 21
\end{align*}

\newpage
\subsection{2.3}

\subsubsection{2.3 (a)}
$Z = (42)_{10}$

\paragraph{Dualdarstellung:}
\begin{align*}
Z = (42)_{10}\\
42: 2 = 21 \mbox{ Rest} &\rightarrow 0\\
21: 2 = 10 \mbox{ Rest} &\rightarrow 1\\
10:2 = 5 \mbox{ Rest} &\rightarrow 0\\
5 : 2 = 2 \mbox{ Rest} &\rightarrow 1\\
2 : 2 = 1 \mbox{ Rest} &\rightarrow 0\\
1 : 2 = 0 \mbox{ Rest} &\rightarrow 1 \uparrow Leserichtung
\end{align*}
$Z = (42)_{10} = (101010)_2$

\paragraph{Oktaldarstellung:}
\begin{align*}
Z &= (101010)_2\\
(010)_2 &= 02\\
(101)_2 &= 05
\end{align*}
(Grupierungen von jeweils 3 Bits)

$Z = (42)_{10} = (52)_8$

\paragraph{Hexadezimaldarstellung:}
\begin{align*}
Z &= (101010)_2\\
(1010)_2 &= \mathtt{A}_{16}\\
(10)_2 &= 2_{16}
\end{align*}
(Grupierungen von jewels 4 Bits)

$Z = (42)_{10} = (2A)_{16}$

\newpage
\subsubsection{2.3 (b)}

$Z = (1969)_{10}$

\paragraph{Dualdarstellung:}
\begin{align*}
1969 : 2 = 984, \mbox{Rest} &\rightarrow 1\\
984 : 2 = 492, \mbox{Rest} &\rightarrow 0\\
492 : 2 = 246, \mbox{Rest} &\rightarrow 0\\
246 : 2 = 123, \mbox{Rest} &\rightarrow 0\\
123 : 2 = 61, \mbox{Rest} &\rightarrow 1\\
61 : 2 = 30, \mbox{Rest} &\rightarrow 1\\
30 : 2= 15, \mbox{Rest} &\rightarrow 0\\
15 : 2 = 7, \mbox{Rest} &\rightarrow 1\\
7 : 2 = 3, \mbox{Rest} &\rightarrow 1\\
3 : 2 = 1, \mbox{Rest} &\rightarrow 1\\
1 : 2 = 0, \mbox{Rest} &\rightarrow 1 \uparrow Leserichtung
\end{align*}

$Z = (1969)_{10} =  (11110110001)_2$

\paragraph{Oktaldarstellung:}
$Z = (11110110001)_2$
\begin{align*}
(001)_2 &= 01\\
(110)_2 &= 06\\
(110)_2 &= 06\\
((0)11)_2 &= 03
\end{align*}
(Grupierungen von jeweils 3 Bits)

$Z = (1969)_{10} = (3661)_{8}$

\paragraph{Hexadezimaldarstellung:}
$Z = (11110110001)_2:$

\begin{align*}
(0001)_2 = 1_{16}\\
(1011)_2 = \mathtt{B}_{16}\\
((0)111)_2= 7_{16}
\end{align*}
(Grupierungen von jewels 4 Bits)

$Z = (1969)_{10} = (\mathtt{7B1})_{16}$

\newpage
\subsubsection{2.3 (c)}
$Z = (5,5625)_{10}$

\paragraph{Dualdarstellung:}
$(5,0)_{10}$:
\begin{align*}
5 : 2 = 2 \mbox{ Rest} &\rightarrow 1\\
2 : 2 = 1 \mbox{ Rest} &\rightarrow 0\\
1 : 2 = 0 \mbox{ Rest} &\rightarrow 1 \uparrow \mbox{Leserichtung}
\end{align*}

$(5)_{10} = (101)_2$\\

$(0,5625)_{10}$:
\begin{align*}
2 \cdot 0,5625 = 1,125 &\rightarrow 1 \downarrow \mbox{Leserichtung}\\
2 \cdot 0,125 = 0,25 &\rightarrow 0\\
2 \cdot 0,25 = 0,5 &\rightarrow 0\\
2 \cdot 0,5 = 1,0 &\rightarrow 1
\end{align*}

$(0,5625)_{10} = (0,1001)_2$\\

$Z = (5,5625)_{10} = (101,1001)_2$

\paragraph{Oktaldarstellung:}
$Z = (101,0)_2$:

\begin{align*}
(101)_2 &= 05
\end{align*}

$Z = (0,1001)_2$:
\begin{align*}
(100_") &= 4\\
(1(00))_2 &= 4
\end{align*}
(Grupierungen von jeweils 3 Bits)

$Z = (5,44)_8$

\paragraph{Hexadezimaldarstellung:}
$Z = (101,0)_2$:
\begin{equation*}
((0)101)_2 = \mathtt{5}
\end{equation*}

$Z = (0,1001)_2$:
\begin{equation*}
(1001)_2 = \mathtt{9}
\end{equation*}

(Grupierungen von jewels 4 Bits)

$Z =(\mathtt{5,9})_{16}$


\subsubsection{2.3 (d)}
$Z = (375,375)_{10}$

\paragraph{Dualdarstellung:}
$Z = (375,0)_{10}:$

\begin{align*}
375 : 2 = 187 Rest &\rightarrow 1\\
187 : 2 = 93 Rest &\rightarrow 1\\
93 : 2 = 46 Rest &\rightarrow 1\\
46 : 2 = 23 Rest &\rightarrow 0\\
23 : 2 = 11 Rest &\rightarrow 1\\
11 : 2 = 5 Rest &\rightarrow 1\\
5 : 2 = 2 Rest &\rightarrow 1\\
2: 2 = 1  Rest &\rightarrow 0\\
1 : 2 = 0 Rest &\rightarrow 1 \uparrow \mbox{Leserichtung}
\end{align*}

$(375,0)_{10}= (101110111)_2$\\

$(0,375)_{10}$:
\begin{align*}
2 \cdot 0,375 = 0,75 &\rightarrow 0 \downarrow \mbox{Leserichtung}\\
2 \cdot 0,75= 1,5 &\rightarrow 1\\
2 \cdot 0,5 = 1,0 &\rightarrow 1
\end{align*}

$(0,375)_{10} = (0,011)_{2}$

$Z = (101110111,011)_2$

\paragraph{Oktaldarstellung:}
$Z_1 = (101110111,0)_2$:
\begin{align*}
(111)_2 &= 07\\
(110)_2 &= 06\\
(101)_2 &= 05
\end{align*}

$Z_2 = (0,011)_2$:
\begin{equation*}
011 = 3
\end{equation*}
(Grupierungen von jeweils 3 Bits)

$Z = (567,3)_8$

\paragraph{Hexadezimaldarstellung:}

$Z = (101110111,0)_2$:
\begin{align*}
(0111)_2 &= \mathtt{7}\\
(0111)_2 &= \mathtt{7}\\
((000)1)_2 &= \mathtt{1}
\end{align*}

$Z = (0,011)_2$:
\begin{equation*}
(011(0))_2 = 6
\end{equation*}

(Grupierungen von jewels 4 Bits)

$Z = (\mathtt{177,6})_{16}$

\subsection{2.4}

\subsubsection{2.4 (a) 1110,1001}
\begin{align*}
1\cdot2^3 + 1\cdot2^2 + 1\cdot2^1 + 0\cdot2^0 + 1\cdot2^{-1} + 0\cdot2^{-2} + 0\cdot2^{-3} + 1\cdot2^{-4}\\
= 14,5625
\end{align*}

\subsubsection{2.4 (b) 10101,11011}
0-Koeffizienten wurden hier weggelassen
\begin{align*}
1\cdot2^4 + 1\cdot2^2 + 1\cdot2^0 + 1\cdot2^{-1} + 1\cdot2^{-2} + 1\cdot2^{-4} + 1\cdot2^{-5}\\
= 21.84375
\end{align*}


\subsection{2.5}
\begin{equation*}
\mathtt{9A,C}_{16} = 9\cdot16^1 + 10\cdot16^0 + 12\cdot16^{-1} = 154.75
\end{equation*}

\newpage

\bibliography{cites}
\bibliographystyle{ieeetr}

\end{document}