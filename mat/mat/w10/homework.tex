\documentclass[11pt,a4paper]{article}

\usepackage{amsmath}
\usepackage{amsfonts}
\usepackage{amssymb}
\usepackage{stmaryrd}
\usepackage[utf8]{inputenc}
\usepackage[ngerman]{babel}
\usepackage{hyperref}
\usepackage{graphicx}
\usepackage{listings}
\usepackage{fancyhdr}
\usepackage{tikz}
\usetikzlibrary{arrows,%
                petri,%
                topaths}%
\usepackage{tkz-berge}
%\usepackage{fullpage}

\title{\textbf{Mathe Hausaufgaben zum 13. Januar 2017}}
\author{Elaha Khaleqi(6947801), Matz Radloff(6946325)}

\pagestyle{fancy}
\fancyhf{}
\rhead{Elaha Khaleqi(6947801), Matz Radloff(6946325) - Mathe I\_Inf}
\cfoot{\thepage}

\newcommand{\Myrightarrow}{\stackrel{\mathclap{\normalfont\mbox{(*)}}}{\Rightarrow}}


\begin{document}
\pagenumbering{arabic}
  \maketitle
  \date{}

\section*{4.}
Nachricht: 5
Öffentlicher Schlüssel: $(11, 247)$
Privater Schlüssel: $(59, 247)$

\begin{align*}
M &= 5\\
5^{11} &= 48828125 = 197684 \cdot 247 + 177\\
\Rightarrow C &= 177 \equiv 5^{11} (\mbox{mod } 247)\\
\end{align*}

Die verschlüsselte Nachricht lautet $C = 177$.
Entschlüsselung zur Probe:

\begin{align*}
177^{59} &= 4.2699492701451408226820636128889... \cdot 10^{132}\\
&= 1.7287244008684780658631836489429... \cdot 10^{130} \cdot 247 + 5\\
\Rightarrow M' &= M = 5
\end{align*}

\section*{5.}

\subsection*{(a)}

\begin{align*}
p &= 7, q = 11, N = 77\\
\varphi(N) &= (p - 1)(q - 1) = 60\\
e &= 13\\
d &= \frac{r \cdot \varphi(N) + 1}{e}\\
\end{align*}
Für ein großes $r = 4615384297$ ergibt sich $d = 999999931$.

Der private Schlüssel lautet also $(999999931, 60)$ und der öffentliche öffentliche $(13, 60)$.
Die Zahlen wurden so gewählt, damit Nachrichten schnell verschlüsselt werden können, aber beim Entschlüsseln von einem möglichen Angreifer, $d$ nicht durch einfaches Ausprobieren erraten werden kann (In praktischen Anwendungsfällen sollten die Zahlen natürlich trotzdem deutlich größer gewählt werden).

\subsection*{(b)}


\end{document}