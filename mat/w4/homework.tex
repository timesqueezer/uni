\documentclass[11pt,a4paper]{article}

\usepackage{amsmath}
\usepackage{amsfonts}
\usepackage{amssymb}
\usepackage{stmaryrd}
\usepackage[utf8]{inputenc}
\usepackage[ngerman]{babel}
\usepackage{hyperref}
\usepackage{graphicx}
\usepackage{listings}
\usepackage{fancyhdr}
%\usepackage{fullpage}

\title{\textbf{Mathe Hausaufgaben zum 17. und 18. November 2016}}
\author{Elaha Khaleqi(6947801), Matz Radloff(6946325)}

\pagestyle{fancy}
\fancyhf{}
\rhead{Elaha Khaleqi(6947801), Matz Radloff(6946325) - Mathe I\_Inf}
\cfoot{\thepage}

\newcommand{\Myrightarrow}{\stackrel{\mathclap{\normalfont\mbox{(*)}}}{\Rightarrow}}


\begin{document}
\pagenumbering{arabic}
  \maketitle
  \date{}

\section{} % 1

\subsection*{(a)}

\paragraph{Behauptung:}
Ein Produkt $a \cdot b$ ganzer Zahlen ist nur dann durch $3$ teilbar, wenn mindestens einer der Faktoren durch $3$ teilbar ist.

\paragraph{Widerspruchsbeweis:}
Wir zeigen, dass das Produkt zweier, nicht durch $3$ teilbare Zahlen auch nicht durch $3$ teilbar sein kann.

\begin{equation*}
3k = a \cdot b; a,b \in \{3n+1, 3n+2\}; k,n \in \mathbb{Z}
\end{equation*}

Es ergeben sich die folgenden drei Fälle ($n, o, k \in \mathbb{Z}$):

\paragraph{$a = 3n+1, b = 3o+1$}

\begin{align*}
    3k &= (3n+1) \cdot (3o+1)\\
    3k &= 9no + 3n + 3o + 1\\
    3k &= 3(3no + n + o) + 1 \lightning
\end{align*}

\paragraph{$a = 3n+1, b = 3o+2$}

\begin{align*}
    3k &= (3n+1) \cdot (3o+2)\\
    3k &= 9no + 6n + 3o + 2\\
    3k &= 3(3no + 2n + o) + 2 \lightning
\end{align*}

\paragraph{$a = 3n+2, b = 3o+2$}
\begin{align*}
    3k &= (3n+2) \cdot (3o+2)\\
    3k &= 9no + 6n + 6o + 4\\
    3k &= 3(3no + 2n + 2o) + 4 \lightning
\end{align*}

Folglich gibt es keine Kombination von nicht durch $3$ teilbaren Zahlen, deren Produkt durch $3$ teilbar ist. \checkmark

\subsection*{(b)}
Wir zeigen, dass $\sqrt{3}$ eine irrationale Zahl ist.

\paragraph{Widerspruchsbeweis:}
Wir nehmen an, $\sqrt{3}$ ließe sich als Bruch $\frac{n}{m}; n, m \in \mathbb{Z}, m \neq 0$ darstellen, der maximal gekürzt sei.

Für den Beweis müssen wir vorher feststellen, ob die Wurzel einer durch $3$ teilbaren Zahl auch durch $3$ teilbar ist:

\begin{align*}
a, k &\in \mathbb{Z}\\
a &= 3k\\
a^2 &= 9k^2 = 3(3k^2) \tag*{\checkmark $(\ast)$}
\end{align*}

\begin{align*}
\sqrt{3} = \frac{n}{m} &\Rightarrow n^2 = 3m^2\\
3 \bigm| 3n^2 &\stackrel{(\ast)}{\Rightarrow} 3 \bigm| n\\
\Rightarrow 9 \bigm| n^2 &\Rightarrow 9k = n^2, k \in \mathbb{Z}\\
9k = 3m^2 &\Rightarrow 3k = m^2\\
\Rightarrow 3 \bigm| m^2 &\stackrel{(\ast)}{\Rightarrow} 3 \bigm| m\\
&\Rightarrow 3 \bigm| m, 3 \bigm| n \lightning
\end{align*}

\section{} % 2
Für zwei Äquivalenzklassen $[(a,b)], [(c,d)] \in \mathbb{Z}$ wurde die Summe $[(a,b)] + [(c,d)]$ als die Äquivalenzklasse $[(a+c,b+d)]$ definiert. Wir zeigen, dass die Addition wohldefiniert ist.
Unter Annahme der Definition aus der Vorlesung, die die Menge $\mathbb{Z}$ als die Faktormenge $(\mathbb{N}_0 \times \mathbb{N}_0) / \sim$ definiert, kann man die Summe wie folgt schreiben:

\begin{align*}
(a+c)+(b+d)' &= (a+c)' + (b+d)\\
(a+c)-(b+d) &= (a+c)' + (b+d)'\\
(a-b)+(c-d) &= (a-b)' + (c-d)'
\end{align*}
Laut Definition ist $(a-b) \Leftrightarrow a+b' = a'+b \Leftrightarrow [(a,b)]$ eine Darstellung für eine Zahl aus $\mathbb{Z}$. Somit hängt die Addition $[(a+c,b+d)]$ nur von den Äquivalenzklassen $[(a,b)]$ und $[(c,d)]$ ab.

\newpage
\section{} % 3
Wir zeigen, dass $\sqrt{6}$ eine irrationale Zahl ist.

\paragraph{Widerspruchsbeweis:}
Wir nehmen an, $\sqrt{6}$ ließe sich als Bruch $\frac{n}{m}; n, m \in \mathbb{Z}, m \neq 0$ darstellen, der maximal gekürzt sei.

\noindent Für den Beweis müssen wir vorher feststellen, ob die Wurzel einer durch $6$ teilbaren Zahl auch durch $6$ teilbar ist:

\begin{align*}
a, k &\in \mathbb{Z}\\
a &= 6k\\
a^2 &= 36k^2
\end{align*}

Hierbei ist durch den Faktor $36$ nicht eindeutig, dass $a^2$ auch durch $6$ teilbar ist. Durch Primfaktorzerlegung ergeben sich z.B. auch die Faktoren $9 \cdot 4$ (Ausgehend von $2^2 \cdot 3^2$). Damit die notwendige Behauptung $6 \bigm| a^2$ trotzdem gilt, teilen wir das Problem auf, indem wir benutzen, dass gilt: $2 \bigm| a \wedge 3 \bigm| a \Rightarrow 6 \bigm| a$
Nun benutzen wir unsere Feststellung aus 1. (b) und Lemma 3.XX aus der Vorlesung, sodass folgt:

\begin{align*}
3 \bigm| a^2 &\wedge 2 \bigm| a^2\\
3 \bigm| a &\wedge 2 \bigm| a
\end{align*}

\begin{align*}
\sqrt{6} = \frac{n}{m} &\Rightarrow n^2 = 6m^2\\
2 \bigm| 3n^2 \wedge 3 \bigm| 3n^2 &\Rightarrow 2 \bigm| n \wedge 3 \bigm| n\\
\Rightarrow 6 \bigm| n &\Rightarrow 36 \bigm| n^2\\
\Rightarrow 9k &= n^2, k \in \mathbb{Z}\\
36k = 6m^2 &\Rightarrow 6k = m^2\\
\Rightarrow 6 \bigm| m^2 \Rightarrow 6 \bigm| m\\
&\Rightarrow 6 \bigm| m, 6 \bigm| n \lightning
\end{align*}


\section{ggT(3213,234):} % 4
\begin{align*}
3213 &= 324 * 13 + 171\\
234 &= 171 * 1 + 63\\
171 &= 63 * 2 + 45\\
63 &= 45 * 1 + 18\\
45 &= 18 * 2 + 9\\
18 &= 9 * 2 + 0\\
\Rightarrow ggt(3213,234) &= 9
\end{align*}

\section{} % 5

\subsection*{(a) Festellung, ob folgende Aussagen wahr oder falsch sind:}

\subsubsection*{i. $a_1 \bigm| b_1 \wedge a_2 \bigm| b_2 \Rightarrow a_1 + a_2 \bigm| b_1 + b_2$}

Damit die Teilbarkeit von $b_1$ und $b_2$ gegeben ist, müssen jeweils $c_1, c_2 \in \mathbb{Z}$ existieren, sodass gilt:
\begin{equation*}
b_1 = c_1 \cdot a_1 \wedge b_2 = c_2 \cdot a_2
\end{equation*}

Außerdem muss für die Teilbarkeit von $b_1 + b_2$ gelten:

\begin{equation*}
b_1 + b_2 = (a_1 + a_2) \cdot d, d \in \mathbb{Z}
\end{equation*}

Durch Umstellen erhält man:

\begin{equation*}
b_1 + b_2 = a_1e + a_2e
\end{equation*}
Folglich müssten, damit die Aussage gilt, $c_1$ und $c_2$ gleich sein.\\
$\rightarrow$ Aussage stimmt nicht.

\subsubsection*{ii. $a \bigm| b_1 \wedge a \bigm| b_2 \Rightarrow a \bigm| b_1 + b_2$}

\begin{align*}
b_1 &= c_1a\\
b_2 &= c_2a\\
b_1 + b_2 &= c_1a + c_2a = a \cdot (c_1 + c_2) \tag*{\checkmark}
\end{align*}

\subsubsection*{iii. $a \bigm| b_1 \wedge a \bigm| b_2 \Rightarrow a \bigm| b_1 - b_2$}

\begin{align*}
b_1 &= c_1a\\
b_2 &= c_2a\\
b_1 - b_2 &= c_1a - c_2a = a \cdot (c_1 - c_2) \tag*{\checkmark}
\end{align*}

\subsection*{(b)}
\begin{align*}
a|b \land b|a; a,b\in\mathbb{N}_0\\
\Rightarrow b=a*n, n\in\mathbb{N}\\
\Rightarrow a=b*m, m\in\mathbb{N}\\
\Rightarrow b=(b*m)*n
=b=b*(m*n)\\
\Rightarrow m*n=1\\
\Rightarrow m=1;n=1\\
\Rightarrow b=a*n=a*1=a
\end{align*}
\begin{flushleft}
    Da es für m und n nur eine mögliche Lösung gibt, ist die Teilbarkeitsrelation auf $\mathbb{N}_0$ antisymmetrisch. Da $1,-1 \in \mathbb{Z}$ und sowohl $1|-1$ als auch $-1|1$, kann die Teilbarkeitsrelation auf $\mathbb{Z}$ nicht antisymmetrisch sein.
\end{flushleft}

\end{document}$