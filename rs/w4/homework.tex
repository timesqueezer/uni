\documentclass[11pt,a4paper]{article}
\usepackage{spreadtab}
\usepackage{amsmath}
\usepackage{amsfonts}
\usepackage[utf8]{inputenc}
\usepackage[ngerman]{babel}
\usepackage{hyperref}
\usepackage{graphicx}
\usepackage{listings}
\usepackage{fancyhdr}
\usepackage{eurosym}
\usepackage{url}
\usepackage{hyperref}
\usepackage[]{units}
%\usepackage{fullpage}

\title{\textbf{Rechnerstrukturen Hausaufgaben zum 16. November 2016}}
\author{Ali Ebrahimi Pourasad, Moritz Lahann, Matz Radloff}

\pagestyle{fancy}
\fancyhf{}
\lfoot{Ali Ebrahimi Pourasad(6948107)\\ Moritz Lahann(6948050)\\ Matz Radloff(6946325)}
\lhead{[RS\_GR1\_12]}
\cfoot{\thepage}


\begin{document}
  \maketitle
  \date{}
  Gruppe: [RS\_GR1\_12]

\section*{4.1}
\label {sec:4.1}

\paragraph{Gegeben:} $A = 10100.1101, B = 10.01$

\subsection*{(a) $A+B$}

\begin{align*}
  10100&.1101\\
+ 00010&.0100\\
= 10111&.0001
\end{align*}

\subsection*{(b) $A-B$}

\begin{align*}
  10100&.1101\\
- 00010&.0100\\
= 10000&.1001
\end{align*}

\subsection*{(c) $A \cdot B$}

\begin{align*}
  10100&.1 1 01 \cdot 10.01\\
  10100&.1 1  01 \cdot 1\\
   0000& 0.0  00 \cdot 0\\
    000& 0.0  000 \cdot 0\\
     10& 1 00.1101 \cdot 1\\
=10111011.0001
\end{align*}
(muss ich nochmal nachrechen)

\subsection*{(c) $A / B$}

\begin{align*}
\end{align*}

\section*{4.2}
Darstellung folgender Zahlen als Gleitkommazahl einfacher Genauigkeit gemäß IEEE 754.

\subsection*{(a) $-3,75$}
Binärdarstellung ohne Vorzeichen: $(11,11)_2$
Für die Normalisierung verschieben wir das Komma eine Stelle nach links; der Exponent vergrößert sich also um $1$.
Inklusive Vorzeichen erhält man:
\begin{itemize}
  \item Vorzeichen: 1
  \item Exponent(Exzess-127): 0000 1010
  \item Mantisse: (1), 1110 0000 0000 0000 0000 000
  \item IEEE 754: 1 0000 1010 1110 0000 0000 0000 0000 000
\end{itemize}
1


\end{document}